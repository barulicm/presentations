\documentclass{beamer}
%\documentclass[handout]{beamer}

\usepackage{pgfpages}
\usepackage{listings}

\mode<handout>{%
	\pgfpagesuselayout{8 on 1}[a4paper] 
	\setbeameroption{show notes}
	\setbeamertemplate{note page}{\huge Notes\\ \normalsize \insertnote}
}

\usetheme{Madrid}

\setbeamertemplate{footline}
{
	\leavevmode%
	\hbox{%
		\begin{beamercolorbox}[wd=.333333\paperwidth,ht=2.25ex,dp=1ex,center]{author in head/foot}%
			\usebeamerfont{author in head/foot}\insertshortauthor%~~\beamer@ifempty{\insertshortinstitute}{}{(\insertshortinstitute)}
		\end{beamercolorbox}%
		\begin{beamercolorbox}[wd=.333333\paperwidth,ht=2.25ex,dp=1ex,center]{title in head/foot}%
			\usebeamerfont{title in head/foot}\insertshorttitle
		\end{beamercolorbox}%
		\begin{beamercolorbox}[wd=.333333\paperwidth,ht=2.25ex,dp=1ex,right]{date in head/foot}%
			\insertframenumber{} / \inserttotalframenumber\hspace*{2ex} 
		\end{beamercolorbox}
		}%
	\vskip0pt%
}
	
\author{Matthew Barulic}

\date{March 15, 2018}

\title{Introduction to C++ Design}

\subtitle{Some fundamental principles for experienced programmers}

\institute{
	Wheego Technologies
}

\setbeamertemplate{navigation symbols}{}%hide navigation symbols

\usecolortheme[named=blue]{structure}

\AtBeginSection[]{
	\begin{frame}
		\vfill
		\centering
		\begin{beamercolorbox}[sep=8pt,center,shadow=true,rounded=true]{title}
			\usebeamerfont{title}\insertsectionhead\par%
		\end{beamercolorbox}
		\vfill
	\end{frame}
}

\begin{document}
	
	\begin{frame}
		\titlepage
	\end{frame}
	
	\begin{frame}{Today’s Topics}
		\tableofcontents
	\end{frame}
	
	\section{Types}
	
	\begin{frame}{Types}
		\begin{center}
			\textbf{C++ typing is strong and static.}
		\end{center}
		\bigskip
		\begin{itemize}
			\item \textbf{Strong} - Types are checked for compatibility in assignment and use.
			\item \textbf{Static} - Expressions and variables cannot change type at runtime.
		\end{itemize}
	\end{frame}
	
	\begin{frame}{Types}
		\begin{center}
			For comparison,\\
			Python typing is strong and dynamic. 
		\end{center}
		\bigskip
		\begin{center}
			See: \href{https://en.wikipedia.org/wiki/Duck_typing}{Duck Typing}
		\end{center}
	\end{frame}
	
	\defverbatim[colored]\lstI{
		\begin{lstlisting}[language=C++,basicstyle=\ttfamily,keywordstyle=\color{red}]
int a = 10;
 ^
 
double sum(list<double>& numbers);
  ^         ^     ^
		\end{lstlisting}
	}
	
	\begin{frame}{Types}
		\begin{center}
		Every named entity in C++ has a type
		\end{center}
		\bigskip
		\begin{exampleblock}{Example}
		\lstI
		\end{exampleblock}
	\end{frame}
	
	\defverbatim[colored]\lstI{
		\begin{lstlisting}[language=C++,basicstyle=\ttfamily,keywordstyle=\color{red}]
using Location3D = std::array<double,3>;
using OrientationRPY = std::array<double,3>;
		\end{lstlisting}
	}
	
	\begin{frame}{Types}
		Tips about types:
		\begin{itemize}
			\item Types should enforce semantically meaningful code.
			\item Type names should be very clear for human readers.
			\item Consider making types even without unique behavior.
		\end{itemize}
		\bigskip
		\begin{exampleblock}{Example}
			\lstI
		\end{exampleblock}
	\end{frame}
	
	\defverbatim[colored]\lstI{
		\begin{lstlisting}[language=C++,basicstyle=\ttfamily,keywordstyle=\color{red}]
MyVeryLongTypeName a = createMyVeryLongTypeName(...);
		\end{lstlisting}
	}
	
	\begin{frame}{auto}
		\begin{center}
			Sometimes, being overly explicit with type names can leave code hard to read.
		\end{center}
		\bigskip
		\begin{exampleblock}{Example}
			\lstI
		\end{exampleblock}
		
	\end{frame}
	
	\defverbatim[colored]\lstI{
		\begin{lstlisting}[language=C++,basicstyle=\ttfamily,keywordstyle=\color{red}]
MyVeryLongTypeName a = createMyVeryLongTypeName(...);
auto b = createMyVeryLongTypeName(...);
auto c; // Does not compile! No type info.
		\end{lstlisting}
	}
	
	\begin{frame}{auto}
		The \textbf{auto} keyword
		\begin{itemize}
			\item Introduced in C++11
			\item Asks the compiler to infer types
			\item Does \textbf{NOT} suddenly make C++ dynamically typed
			\item Types must be known (if only to the compiler) at compile time
		\end{itemize}
		
		\begin{exampleblock}{Example}
		\lstI
		\end{exampleblock}
		
	\end{frame}
	
	\defverbatim[colored]\lstI{
		\begin{lstlisting}[language=C++,basicstyle=\ttfamily,keywordstyle=\color{red}]
std::vector<int> numbers = ...

for(std::vector<int>::iterator iter = numbers.begin();
  iter != numbers.end();
  iter++)
{ ... }

for(auto iter = numbers.begin();
  iter != numbers.end();
  iter++)
{ ... }
		\end{lstlisting}
	}
	
	\begin{frame}{auto}
		\textbf{auto} shows up in a lot of code because it can clean up redundant type specifications from the human reader's point of view while still explicitly conveying type information to the compiler.
		
		\begin{exampleblock}{Example}
			\lstI
		\end{exampleblock}
		
	\end{frame}
	
	\begin{frame}{Templates}
		\begin{center}
			Sometimes, we genuinely don't know the\\ 
			type of an object when we \textbf{write} our code.\\
			\bigskip
			However, the compiler will know all of\\ 
			the types when we \textbf{compile} our code.
		\end{center}
	\end{frame}
	
	\defverbatim[colored]\lstI{
		\begin{lstlisting}[language=C++,basicstyle=\ttfamily,keywordstyle=\color{red}]
template<typename T>
T add(const T& a, const T& b) {
  return a + b;
}
		\end{lstlisting}
	}
	
	\begin{frame}{Templates}
		Templated entities...
		\begin{itemize}
			\item Use a placeholder type to be filled at compile time.
			\item Are duplicated for each unique set of template parameters used.
			\item Cannot be extended with new instantiations at runtime.
		\end{itemize}
		\bigskip
		\begin{exampleblock}{Example}
			\lstI
		\end{exampleblock}
	\end{frame}
	
		\defverbatim[colored]\lstI{
			\begin{lstlisting}[language=C++,basicstyle=\ttfamily,keywordstyle=\color{red}]
template<typename T>
T add(const T& a, const T& b) {
  return a + b;
}

auto a = 10;
auto b = 5;
auto c = 7.0;

add<int>(a,b); // OK, a & b are both int's
add<int>(a,c); // OK, c is cast to int
add<>(a,b); // OK, a & b are both int's
add<>(a,c); // BAD, int or double?
			\end{lstlisting}
		}
		
		\begin{frame}{Templates}
			\begin{exampleblock}{Example}
				\lstI
			\end{exampleblock}
		\end{frame}
	
	\defverbatim[colored]\lstI{
	\begin{lstlisting}[language=C++,basicstyle=\ttfamily,keywordstyle=\color{red}]
auto distance = 30.0 * meter;
auto elapsed = 10.0 * second;
std::cout << (distance / elapsed) << "\n";
// Output: 3 m s^-1
	\end{lstlisting}
	}
	
	\begin{frame}{Example: Boost.Units}
		\begin{itemize}
			\item Types facilitate semantic operations
			\item Boost.Units is a library for dimensional analysis
		\end{itemize}
		
		\begin{exampleblock}{Example}
		\lstI
		\end{exampleblock}
		
		\bigskip
		\bigskip
		
		\begin{center}
			\tiny{\url{http://www.boost.org/doc/libs/1_65_0/doc/html/boost_units/Quick_Start.html}}
		\end{center}
	\end{frame}
	
	\begin{frame}{Types Recap}
		\begin{itemize}
			\item Types facilitate semantically meaningful code.
			\medskip
			\item Types are enforced at compile time.
			\medskip
			\item \textbf{auto} allows us to have clean code without losing types.
			\medskip
			\item Templates allow us to share code and behavior across types.
		\end{itemize}
	\end{frame}
	
	\section{Objects and Lifetimes}
	
	\begin{frame}
		\begin{center}
			Why C++?
		\end{center}
	\end{frame}
	
	\begin{frame}
		\begin{center}
			Why C++?\\
			\bigskip
			\textbf{Deterministic Object Lifetimes}
		\end{center}
	\end{frame}
	
	\begin{frame}
		\begin{center}
			A brief tangent about lightsabers...
			\includegraphics[width=\textwidth]{images/lightsabers}
		\end{center}
	\end{frame}
	
	\defverbatim[colored]\lstI{
		\begin{lstlisting}[language=C++,basicstyle=\ttfamily,keywordstyle=\color{red}]
int main() {
  Robot robot("Robby");
  // ^ Constructor call begins object lifetime
  ...
  return 0;
}
// ^ End of function ends object lifetime
		\end{lstlisting}
	}
	
	\begin{frame}{Lifetimes}
		Object lifetimes...
		\begin{itemize}
			\item Begin after initialization.
			\item End before destruction.
			\item Are usually based on variable scope.
		\end{itemize}
		\bigskip
		\begin{exampleblock}{Example}
			\lstI
		\end{exampleblock}
	\end{frame}
	
	\defverbatim[colored]\lstI{
		\begin{lstlisting}[language=C++,basicstyle=\ttfamily,keywordstyle=\color{red}]
class Robot {
  Robot(std::string name)
  { ... }
  
  Robot()
  : Robot("Unnamed")
  { ... }
};
		\end{lstlisting}
	}
	
	\begin{frame}{Constructors}
		Construcotrs...
		\begin{itemize}
			\item Initialize an object.
			\item Are analogous to Python's \texttt{\char`_\char`_init\char`_\char`_()} special functions.
		\end{itemize}
		\bigskip
		\begin{exampleblock}{Example}
			\lstI
		\end{exampleblock}
	\end{frame}
	
	\defverbatim[colored]\lstI{
		\begin{lstlisting}[language=C++,basicstyle=\ttfamily,keywordstyle=\color{red}]
Robot robot;
// ^ Calls Robot's no-args constructor.
// Compilation fails if Robot doesn't have one.
		\end{lstlisting}
	}
	
	\begin{frame}{Constructors}
		\centering\textit{Objects} cannot be left uninitialized.
		\bigskip
		\begin{exampleblock}{Example}
			\lstI
		\end{exampleblock}
	\end{frame}
	
	\defverbatim[colored]\lstI{
		\begin{lstlisting}[language=C++,basicstyle=\ttfamily,keywordstyle=\color{red}]
class TCPConnection {
  TCPConnection()
  { ... } // Open network socket

  ~TCPConnection()
  { ... } // Close network socket
};
		\end{lstlisting}
	}
	
	\begin{frame}{Destructors}
		Destructors...
		\begin{itemize}
			\item Cleanup any resources owned by the object.
			\item Are comparable to Python's \texttt{\char`_\char`_del\char`_\char`_()} special functions.
		\end{itemize}
		\bigskip
		\begin{exampleblock}{Example}
			\lstI
		\end{exampleblock}
	\end{frame}
	
	\begin{frame}{RAII}
		\begin{center}
			\huge RAII\\
			\bigskip
			\normalsize\textbf{R}esource \textbf{A}cquisition \textbf{I}s \textbf{I}nitialization
		\end{center}
	\end{frame}
	
	\begin{frame}{RAII}
		\begin{itemize}
			\item Required resources should be acquired during construction.
			\medskip
			\item Acquired resources should be released during destruction.
			\medskip
			\item Eliminates the need for explicit \texttt{open()} or \texttt{close()} methods.
			\medskip
			\item Significantly reduces the likelihood of resource mismanagement.
		\end{itemize}
	\end{frame}
	
	\defverbatim[colored]\lstI{
		\begin{lstlisting}[language=Python,basicstyle=\ttfamily,keywordstyle=\color{red}]
context = zmq.Context()
publisher = ...
...
pulbisher.close()
context.term()
		\end{lstlisting}
	}
	
	\defverbatim[colored]\lstII{
		\begin{lstlisting}[language=C++,basicstyle=\ttfamily,keywordstyle=\color{red}]
zmq::Context context;
auto publisher = ...
...
		\end{lstlisting}
	}
	
	\begin{frame}{RAII}
		\begin{exampleblock}{Example - Python}
			\lstI
		\end{exampleblock}
		\begin{exampleblock}{Example - C++}
			\lstII
		\end{exampleblock}
	\end{frame}
	
	\begin{frame}{Objects and Lifetimes Recap}
		\begin{itemize}
			\item C++ maintains clearly defined boundaries on object lifetimes.
			\medskip
			\item Constructors and destructors give control over either end of an objects lifetime.
			\medskip
			\item RAII design takes advantage of lifetime mechanics to safely manage resources.
		\end{itemize}
	\end{frame}

\end{document}
